% Options for packages loaded elsewhere
\PassOptionsToPackage{unicode}{hyperref}
\PassOptionsToPackage{hyphens}{url}
%
\documentclass[
]{article}
\usepackage{amsmath,amssymb}
\usepackage{lmodern}
\usepackage{ifxetex,ifluatex}
\ifnum 0\ifxetex 1\fi\ifluatex 1\fi=0 % if pdftex
  \usepackage[T1]{fontenc}
  \usepackage[utf8]{inputenc}
  \usepackage{textcomp} % provide euro and other symbols
\else % if luatex or xetex
  \usepackage{unicode-math}
  \defaultfontfeatures{Scale=MatchLowercase}
  \defaultfontfeatures[\rmfamily]{Ligatures=TeX,Scale=1}
\fi
% Use upquote if available, for straight quotes in verbatim environments
\IfFileExists{upquote.sty}{\usepackage{upquote}}{}
\IfFileExists{microtype.sty}{% use microtype if available
  \usepackage[]{microtype}
  \UseMicrotypeSet[protrusion]{basicmath} % disable protrusion for tt fonts
}{}
\makeatletter
\@ifundefined{KOMAClassName}{% if non-KOMA class
  \IfFileExists{parskip.sty}{%
    \usepackage{parskip}
  }{% else
    \setlength{\parindent}{0pt}
    \setlength{\parskip}{6pt plus 2pt minus 1pt}}
}{% if KOMA class
  \KOMAoptions{parskip=half}}
\makeatother
\usepackage{xcolor}
\IfFileExists{xurl.sty}{\usepackage{xurl}}{} % add URL line breaks if available
\IfFileExists{bookmark.sty}{\usepackage{bookmark}}{\usepackage{hyperref}}
\hypersetup{
  pdftitle={02427 Assignment 3},
  pdfauthor={August Thomas Hjortshøj Schreyer s163716},
  hidelinks,
  pdfcreator={LaTeX via pandoc}}
\urlstyle{same} % disable monospaced font for URLs
\usepackage[margin=1in]{geometry}
\usepackage{graphicx}
\makeatletter
\def\maxwidth{\ifdim\Gin@nat@width>\linewidth\linewidth\else\Gin@nat@width\fi}
\def\maxheight{\ifdim\Gin@nat@height>\textheight\textheight\else\Gin@nat@height\fi}
\makeatother
% Scale images if necessary, so that they will not overflow the page
% margins by default, and it is still possible to overwrite the defaults
% using explicit options in \includegraphics[width, height, ...]{}
\setkeys{Gin}{width=\maxwidth,height=\maxheight,keepaspectratio}
% Set default figure placement to htbp
\makeatletter
\def\fps@figure{htbp}
\makeatother
\setlength{\emergencystretch}{3em} % prevent overfull lines
\providecommand{\tightlist}{%
  \setlength{\itemsep}{0pt}\setlength{\parskip}{0pt}}
\setcounter{secnumdepth}{-\maxdimen} % remove section numbering
\usepackage{bm}
\usepackage{todonotes}
\usepackage[noabbrev,nameinlink,capitalise]{cleveref}
\usepackage{float}
\usepackage{placeins}
\ifluatex
  \usepackage{selnolig}  % disable illegal ligatures
\fi

\title{02427 Assignment 3}
\author{August Thomas Hjortshøj Schreyer s163716 \and }
\date{xx/03/2021}

\begin{document}
\maketitle

{
\setcounter{tocdepth}{2}
\tableofcontents
}
\hypertarget{contribution-to-the-assignment}{%
\section{Contribution to the
Assignment}\label{contribution-to-the-assignment}}

We made our best attempt to ensure equal contribution to the making of
this assignment. \FloatBarrier

\hypertarget{conclusion}{%
\subsection{Conclusion}\label{conclusion}}

\FloatBarrier

\hypertarget{problem-a}{%
\section{Problem A}\label{problem-a}}

A general linear model predicting the level of clothing (clo) using
outdoor temperature, indoor operating temperature and sex of the subject
as exploratory variables will be fitted. The full model using all
variables and interactions are used as a starting point, hereafter the
model will be reduced using the backward selection with Type III
partitioning. To obtain type III partitioning partial tests are computed
for all factors, and hence compute the deviance for the variables as
they entered the model last. Higher order variables are removed first if
these are insignificant.

The result from the backward selection using Type III partitioning are
given below:

\[Clo_i =\beta_0 + \beta_1t_{Out,i} + \beta_2t_{InOp,i} + \beta_3sex_i + \beta_4t_{InOp}:sex_i + \epsilon_i\]

where \[\epsilon_i ~ N(0, \sigma^2)\]

BØR VI IKKE BESTEMME SIGMA? SKAL VI SAMMENLIGNE MED H0?

Here : indicates the interaction between the indoor operating
temperature and gender of the subject. The coefficients and confidence
intervals can be seen in table 1.

\begin{table}[ht]
\centering
\begin{tabular}{rrrr}
  \hline
 & 2.5% & Estimate & 97.5% \\ 
  \hline
(Intercept) & 1.51 & 2.13 & 2.76 \\ 
  tOut & -0.02 & -0.01 & -0.01 \\ 
  tIn & -0.07 & -0.05 & -0.02 \\ 
  sexmale & -2.16 & -1.28 & -0.40 \\ 
  tIn:sexmale & 0.01 & 0.04 & 0.08 \\ 
   \hline
\end{tabular}
\end{table}

\FloatBarrier

\hypertarget{problem-b}{%
\section{Problem B}\label{problem-b}}

As we saw from the residuals in Problem A, some of the individuals do
produce residuals that are either all positive or all negative. This
could indicate that the individuals could have different base
preferences of clothing. That is, one individual might be be warm most
of the time and therefore wear less clothing. Likewise an individual
might often be cold and would therefore wear more clothes. This could
indicate that our model should have an individual intercept.

We cannot estimate any models with both an interaction between both of
the indoor/outdoor temperatures and an individual intercept because of
our degrees of freedom. Third order interactions are therefore also not
possible. We can however estimate a model with an interaction between
either the indoor or outdoor temperature and the individuals. This will
lead to either of the following models:

\begin{equation}\label{eq:prob2_full1}
\begin{aligned}
y_i=&\beta_0+a\cdot \beta_1 \cdot t_{in,i}+\beta_2 \cdot t_{out,i}+\beta_3\cdot t_{in,i}\cdot t_{out,i}+a(sex_i)+b(t_{in,i},sex_i)+c(t_{in,i},sex_i)+d(individual_i)+e(t_{in,i},individual_i)\\
&\gamma(t_{out,i},sex_i)+
\end{aligned}
\end{equation}

\begin{equation}\label{eq:prob2_full2}
\begin{aligned}
y_i=\mu+a\cdot b \cdot t_{in,i}+c \cdot t_{out,i}+d\cdot t_{in,i}\cdot t_{out,i}+\alpha(sex_i)+\beta(t_{in,i},sex_i)+\beta(t_{in,i},sex_i)\\
y_i=\mu+a\cdot b \cdot t_{in,i}+c \cdot t_{out,i}+d\cdot t_{in,i}\cdot t_{out,i}+\alpha(sex_i)+\gamma(t_{out,i},sex_i)
\end{aligned}
\end{equation}

The following table shows the initial models with either an interaction
between the indoor/ temperature and the individuals:

\begin{table}[H]
\centering
\begin{tabular}{ccccc}
  \hline
 & Sum Sq & Df & F value & Pr($>$F) \\ 
  \hline
(Intercept) & 0.022 & 1 & 3.386 & 0.073 \\ 
  tOut & 0.023 & 1 & 3.446 & 0.071 \\ 
  tInOp & 0.017 & 1 & 2.531 & 0.120 \\ 
  subjId & 0.480 & 46 & 1.571 & 0.074 \\ 
  tOut:tInOp & 0.023 & 1 & 3.417 & 0.072 \\ 
  tOut:subjId & 0.442 & 46 & 1.449 & 0.117 \\ 
  Residuals & 0.266 & 40 &  &  \\ 
  \hline
  (Intercept)1 & 0.002 & 1 & 0.270 & 0.606 \\ 
  tOut1 & 0.015 & 1 & 1.695 & 0.200 \\ 
  tInOp1 & 0.004 & 1 & 0.417 & 0.522 \\ 
  subjId1 & 0.370 & 46 & 0.925 & 0.603 \\ 
  tOut:tInOp1 & 0.019 & 1 & 2.176 & 0.148 \\ 
  tInOp:subjId & 0.360 & 46 & 0.899 & 0.639 \\ 
  Residuals1 & 0.348 & 40 &  &  \\ 
   \hline
\end{tabular}
\caption{The first rows (until the horizontal line) significant parameters for a model with an interaction term between the outdoor temperature and the individuals. Below the horizontal line we have a model with interaction term between the indoor temperature and the indivuduals. We can see that none of these interaction terms between the individuals and the temperature are significant. Furthermore we see that those are the first terms to be dropped. We used a type 3 for the model deviance table.}
\label{tab:prob2_init}
\end{table}

\cref{tab:prob2_init} shows that neither of the interaction terms
between the temperatures and the individuals are significant. Therefore
we can continue with finding our optimal model.

\FloatBarrier

\hypertarget{problem-c}{%
\subsection{Problem C}\label{problem-c}}

\FloatBarrier

\hypertarget{references}{%
\section{References}\label{references}}

\begin{itemize}
  \item[[1]] Henrik Madsen and Jan Holst, Modelling Non-Linear and Non-Stationary Time Series, December 2006. 
  \item[[2]] Madsen, H. (2007). Time series analysis. CRC Press.
  \item[[3]] Bacher, P. and Madsen, H. (2011). Identifying suitable models for the heat dynamics of buildings. Ernegy and Buildings Volume 43. 
\end{itemize}

\hypertarget{appendix-r-code-in-seperate-file.}{%
\section{Appendix: R-code in seperate
file.}\label{appendix-r-code-in-seperate-file.}}

\end{document}
